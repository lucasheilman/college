%/* ----------------------------------------------------------- */
%/*                                                             */
%/*                          ___                                */
%/*                       |_| | |_/   SPEECH                    */
%/*                       | | | | \   RECOGNITION               */
%/*                       =========   SOFTWARE                  */ 
%/*                                                             */
%/*                                                             */
%/* ----------------------------------------------------------- */
%/*         Copyright: Microsoft Corporation                    */
%/*          1995-2000 Redmond, Washington USA                  */
%/*                    http://www.microsoft.com                */
%/*                                                             */
%/*   Use of this software is governed by a License Agreement   */
%/*    ** See the file License for the Conditions of Use  **    */
%/*    **     This banner notice must not be removed      **    */
%/*                                                             */
%/* ----------------------------------------------------------- */
%
% HTKBook - Steve Young  11/11/95
%

\newpage
\mysect{HList}{HList}

\mysubsect{Function}{HList-Function}

\index{hlist@\htool{HList}|(}
This program will list the contents of one or more
data sources in any \HTK\ supported format.  It uses the full \HTK\ speech
input facilities described in chapter~\ref{c:speechio} and it can thus
read data from a waveform file, from a parameter file and direct from
an audio source.  \htool{HList} provides
a dual r\^{o}le in \HTK.  Firstly, it is used for examining the contents
of speech data files.  For this function, the \texttt{TARGETKIND} configuration
variable should not be set since no conversions
of the data are required.
Secondly, it is used for checking that input 
conversions are being performed properly.  In the latter case, a configuration
designed for a recognition system can be used with \htool{HList} to
make sure that the translation from the source data into the required
observation structure is exactly as intended.  To assist  this, options
are provided to split the input data into separate data streams (\texttt{-n})
and to explicitly list the identity of each parameter in an observation
(\texttt{-o}).

\mysubsect{Use}{HList-Use}

HList is invoked by typing the command line
\begin{verbatim}
   HList [options] file ...
\end{verbatim}
This causes the contents of each {\tt file} to be listed to the
standard output.  If no files are given and the source format
is \texttt{HAUDIO}, then the audio source is listed.  The source
form of the data can be converted and listed in a variety of target forms by
appropriate settings of the configuration variables, in particular
\texttt{TARGETKIND}\footnote{The \texttt{TARGETKIND} is equivalent to
the \texttt{HCOERCE} environment variable used in earlier versions
of HTK}.

The allowable options to \htool{HList} are

\begin{optlist}

  \ttitem{-d} Force each observation to be listed as discrete VQ symbols.
    For this to be possible the source must be either \texttt{DISCRETE}
    or have an associated VQ table specified via the \texttt{VQTABLE}
    configuration variable.

  \ttitem{-e N}  End listing samples at sample index N. 

  \ttitem{-h}  Print the source header information.

  \ttitem{-i N}  Print N items on each line.

  \ttitem{-n N}  Display the data split into N independent data streams.

  \ttitem{-o}  Show the observation structure.  This identifies the r\^{o}le
      of each item in each sample vector.

  \ttitem{-p}  Playback the audio.  When sourcing from an audio device,
     this option enables the playback buffer so that after displaying the
     sampled data, the captured audio is replayed.

  \ttitem{-r}  Print the raw data only.  This is useful for exporting a file
     into a program which can only accept simple character format data.

  \ttitem{-s N}  Start listing samples from sample index N.  The first sample
index is 0.
 
  \ttitem{-t}  Print the target header information.

\stdoptF

\end{optlist}
\stdopts{HList}

\mysubsect{Tracing}{HList-Tracing}

\htool{HList} supports the following trace options where each
trace flag is given using an octal base
\begin{optlist}
   \ttitem{00001} basic progress reporting.
\end{optlist}
Trace flags are set using the \texttt{-T} option or the  \texttt{TRACE} 
configuration variable.
\index{hlist@\htool{HList}|)}


%%% Local Variables: 
%%% mode: latex
%%% TeX-master: "../htkbook"
%%% End: 
